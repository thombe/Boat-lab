

PART II
p2p1
Bruker verdiene 
Kp=15.61
Kd = 9.77

Roterer med klokken pga friksjon i systemet som skaper angular Momentum
Relativt rask og fin i responsen
Prøver å ligge sånn som den ligger på bordet, med pitch reference = 0; 
Når ikke helt. P_t->inf = 0.05; 




p2p2
Nå er vi på HELI1 så vi har K_rp < 0 i stedet for >0 slik som vi har på Heli2

Vi skal velge K_rp slik at responsen er «fast and accurate» 
Vi prøver litt forskjellige verdier. 
Begynner med -20 og ser at den er alt for høy. Får stående svingninger i pitch
Setter ned til -5 og senker gradvis ned. Ser at responsen blir mer og mer rolig og fin, men også treigere. 
Ender med å sette K_rp = -4 
Den overshooter litt, sånn at vi vet at den er i høyeste laget, men den er også ganske rask og accurate. Lavere verdier ga for treig, og høyere verdier ga for lite accurate og for mye svingninger. 



PART III



Denne fungerer ganske bra

___

Hvordan tune Q og R

Q = ( 30, 5, 100)
Fordi vi bryr oss mye mer om pitch enn pitch_rate pga pitch er faktisk vår output 
Vi bryr oss også mye om elevation rate, mer enn pitch, fordi når vi endrer pitch så kommer vi til å miste elevation, om ikke vi skrur opp vektingen på elevation rate mer enn pitch

R:  Begge pådragene er like «dyre» 0.7*ones(2) 

Pådragene koster oss litt for å unngå høyfrekvent vibrasjoner, som vi får dersom R = 0.1*ones(2) 


Vi når ikke helt referansen. Tipper dette er fordi vi ikke har I-ledd

__


P3p2 elev_rate plot

Kan endre på cutoff freq for elevation rate: 50->5 fordi for mye høyfrekvent støy på plottet, for å få bedre plot. 
Men vi plotter slik den er sånn som vi bruker når vi bruker den som regulator. (50s/s+50)

Ser at vi går ut av den lineariserte modellen når vi går for høyt. Den prøver å gå oppover når vi setter på en step i elev rate. Den går oppover en stund, men så når den ikke lenger går rett opp, altså når den blir skrå, må vi ha masse pådrag for å fortsette å ha positiv elev_rate. Det blir for dyrt i forhold til R matrisen. 
Derfor når vi ikke referansen. Den prøver en stund, og så dabber den av, og elev rate går til null. 

_______________________________

P3p3 - Gjør hele oppgaven på ny
Lagrer dette som P3p3 og har de gamle filene også lagret under nytt navn: P3p3_old_not_functional ens

R = 0.7
Q = (30, 5, 100, 1, 1) 
Ser at med integralvirkning så klarer vi feks å holde en elev_rate = 0 når vi ikke er i likevektsposisjon. 
Men ble mye treigere!! Pitch ble veldig veldig treig

Tuner Q for å få bedre respons
Q = (30, 5, 100, 10, 1) 
Ser at vi får litt raskere respons for pitch, men fortsatt mye treigere enn det vi ønsker. 

Vi setter på et Scope på Gamma, for vi lurer på om vi har integrert opp noe på forhånd før pitch_control_start_time
Det ga ikke mye effekt

Q = (30, 5, 50, 100, 100)
Ble litt oscillatorisk og ustabil når vi prøvde å nå en elev_rate_ref som var positiv lenge. MYYE pådrag

Q = (25, 5, 20, 100, 100)
Ble litt bedre og raskere.

Tar nå og sammenligner eigs(A-BK) for det p3p2 og p3p3 or å se hvorfor vi får så treig respons. Men de er ca på samme størrelsesorden, så systemene burde være ca like raske. Ish i hvert fall 


Q = (25,5,20,100,100)
R = 0.1 * ones(2)
Fikk nå at den elev økte og økte. -> setter på Scope for referansen fra Joysticken, som vi ser ikke er helt null. Da ser vi at den av og til ikke helt går tilbake til referansen sin. Det er et problem, men ikke noe vi kan gjøre noe med, fordi vi kan ikke trekke fra et konstantledd 


Okei, så denne her var ganske bra. 

Q = (25, 10, 20, 100, 100)
R = 0.1 *ones(2)
Setter pitch_control_start_time = 0, for den skapte litt problemer. 

Da fungerer den egentlig veldig bra!
Fjerner V_s*, fordi den trengs ikke når vi har integralvirkning. 

Da ender vi på disse

Tegner plots for disse. 


Endrer dead zone til +/. 0.15 for å hindre at vi får konstant feil fra joysticken. 




PART IV

P4p1

P4p2
Okei. Den var ganske bra. Vi har nå kjørt polene ganske langt til venstre
q = (-60, -60, -40, -40, -20, -20) 
Det gjør vi for å få rask respons. Det estimerte systemet er mye raskere enn det fysiske. 
Hvis vi tar de for langt til venstre vil vi få for mye støy. 
Vi kan lage forskjellige plots ved forskjellige q. 

Og så høypassfiltrerer vi elev_rate for å fjerne et konstant avvik som gav oss noe drit. 

Prøver å plotte alle statene
Når vi har plottet alle statene
plot(States.signals.values(;,1:4));
^
    Legger på en pitch ref step 0.4 på 11 sek for plottene 

P4p2_integral 

 q = (-60, -60, -40, -40, -20, -20) 
Setter på et step og step down I elev rate ref. 
Vi ser at vi får et avvik i elev_rate_estimate og elev_rate, hvor estimatet var 0.1 høyere. Dette høypassfiltrerer vi vekk. Fordi det er et konstant avvik, funker det. 
Vi ser at det da er ca ok, men ikke like bra som med state feedback. 
Fortsetter oppover eller nedover etter at man slipper joysticken.


P4p3 
Test the observer and explain why some state estimates are poor. 


I vår lineære modell er elevation helt separat fra pitch. Så vi måler bare travel og prøver å gjette pitch ut i fra det. Det er vanskelig fordi modellen stemmer ikke helt. Når pitch blir høyere gir den heller ikke travel helt lineært. Travel er heller ikke uavh. Av elevation.
 I tillegg til fysisk friksjon som gir travel. 

Men funker ett ganske bra på HELI2. 